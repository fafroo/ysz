%        File: EIS_capacitance.tex
%     Created: Fri Mar 13 10:00 am 2020 C
% Last Change: Fri Mar 13 10:00 am 2020 C
%
\documentclass[a4 paper]{article}
%\usepackage[]{amsmath}
%\usepackage{amssymb}
\usepackage{mathtools}
\usepackage[version=4]{mhchem}
\usepackage[]{SIunits}
\usepackage[]{hyperref}

\newcommand\F{\textrm{F}}
\newcommand\e{\textrm{e}}
\newcommand\Ref{\mathit{ref}}

\newcommand\kB{k_\mathrm{B}}

\newcommand\Ox{\mathrm{O}}
\newcommand\oo{{\ce{O2}}}
\newcommand\nn{{\ce{N2}}}
\newcommand\Om{\mathrm{Om}}
%\newcommand\Ox{\mathrm{Ox}}
\newcommand\Oi{\mathrm{Oi}}
\newcommand\Zr{\mathrm{Zr}}
\newcommand\Yi{\mathrm{Y }}

\newcommand\Mp{\mathrm{M^+}}
\newcommand\eM{\mathrm{e^-}}


\newcommand\MLC{M_\mathrm{C}^\#}
\newcommand\MLA{M_\mathrm{A}^\#}
\newcommand\nLA{n_\mathrm{A}^\#}
\newcommand\nLC{n_\mathrm{C}^\#}
\newcommand\nC{n_\mathrm{C}}
\newcommand\nG{n^\mathrm{g}}
\newcommand\nL{n^\#}
\newcommand\zL{z^\#}
\newcommand\zC{z^\mathrm{C}}
\newcommand\zA{z^\#}
\newcommand\mL{m^\#}
\newcommand\vL{v^\#}
\newcommand\nuL{\nu^\#}
\newcommand\aL{a^\#}
\newcommand\DpR{\Delta \psi_\textrm{R}}
\newcommand\pD{\tilde{D}}
\newcommand\PF{P}
\newcommand\DGR{\Delta G_R}
\newcommand\DGA{\Delta G_A}
\newcommand{\dd}{\mathrm{d}}
\newcommand{\us}[1]{\underset{\textrm{s}}{#1}{}}
\DeclareMathOperator{\sign}{sign}
\begin{document}

\section{Measurement}
A voltage protocol $U(t)$ is applied and a current response $I(t)$ is measured.
\section{Capacitance}
% For a certain voltage range $[-R,R]$ the following voltage protocol is given
% \begin{align}
%     U(t) = \text{linear triangular wave}
% \end{align}
% 
The capacitance $C$ is the change of the charge w.r.t.\ voltage.
For a sufficiently slow change of voltage U(t) we can write
\begin{align}
    I(t) = \frac{\dd Q}{\dd t} = \frac{\dd Q}{\dd U} \frac{\dd U}{\dd t} \approx C(U) v_{\text{rate}}~.
    \label{CAP01}
\end{align}
Capacitance can be reliably defined only for a blocking electrode, since any current that does decay in a stationary state blows it up.
\section{Impedance spectroscopy}
For a certain frequency $\omega$ and voltage amplitude $A$ the protocol
\begin{align}
    U(t) = A \sin \left( \omega t \right) 
\end{align}
is applied. The measured current response is assumed to take form of
an amplified and shifted harmonic wave
\begin{align}
    I(t) = B \sin \left( \omega t  + \omega_0 \right) ~.
\end{align}

\section{Current functional}
Assume the surface balances in the following form, i.e. adsorption is treated as a reaction,
\begin{align}
    \frac{\dd}{\dd t}\us n_\alpha = \sum_r \gamma^r_\alpha R^r 
    \label{}
\end{align}
\begin{align}
    \varepsilon\frac{\dd}{\dd t}\nabla\varphi\nu|_S 
    %
    &= \varepsilon\frac{\dd}{\dd t}\nabla\varphi\nu|_B - \frac{\dd}{\dd t}\int_S^B \varepsilon \Delta \varphi\\
    &= \varepsilon\frac{\dd}{\dd t}\nabla\varphi\nu|_B - \int_S^B \frac{\dd}{\dd t} n^F\\
    &= \varepsilon\frac{\dd}{\dd t}\nabla\varphi\nu|_B + e_0 z_\Om\left[\mathbf{j}_\Om\nu\right]_S^B
    \label{}
\end{align}
The current functional can be then re-written as
\begin{align}
    \frac{I(t)}{A} &=  - e_0\frac{\dd}{\dd t}\left( 
                                                  \sum_{\alpha\neq\eM} z_\alpha\us n_\alpha
                                              \right) 
                       - e_0 z_{\eM}\sum_r \gamma^r_{\eM}R^r 
                      + \varepsilon\frac{\dd}{\dd t}\nabla\varphi\nu|_S\\
                      %
                   &= - e_0\sum_r R^r \underbrace{\sum_\alpha z_\alpha \gamma^r_\alpha}_{=0} 
                      +\varepsilon\frac{\dd}{\dd t}\nabla\varphi\nu|_B 
                      + e_0 z_\Om\left[\mathbf{j}_\Om\nu\right]_S^B\\
                   %
                   &= \varepsilon\frac{\dd}{\dd t}\nabla\varphi\nu|_B 
                   +e_0 z_\Om\left(%
                       - \underbrace{\mathbf{j}_\Om\nu|_S}_{\mathclap{\text{adsorption}}}
                       + \underbrace{\mathbf{j}_\Om\nu|_B}_{\mathclap{\text{bulk flux}}}
                   \right)
    \label{}
\end{align}
Hence the current depends on the adsorption flux, bulk flux and the time derivative of electric field.
Concretely, 
\begin{align}
    \frac{I}{A} = \frac{\widetilde{I}}{A}\left(
                      y|_S,\
                      \us y,\
                      y|_B,\
                      \nabla y\cdot \nu|_B,
                      \nabla \varphi\cdot \nu|_B,
                      \frac{\dd}{\dd t}\nabla \varphi\cdot \nu|_B
                  \right)
    \label{}
\end{align}
\subsection{Current model}
Let define the adsorption reaction as 
\begin{align}
    \ce{O^2-(b) -> O^2-(s)}~.
    \label{}
\end{align}
\begin{align}
    \mathbf{j}_\Om|_S\cdot\nu &= \us A_0 \left(K_A y|_S (1-\us y) -K_A^{-1} (1-y|_S)\us y  \right), \\
    \mathbf{j}_\Om|_B\cdot\nu &= - D \frac{(1 - \nuL)m}{\vL}
                        \left( 1 +m_\Om\frac{(1 - \nuL)m}{\mL} y \right)^2 
                    \Bigg[
                      \frac{\nabla y}{(1- y)}
                      +y \frac{z_{\Om} e_0}{\kB T} \nabla \varphi
                    \Bigg]\cdot\nu\Bigg|_B,
    \label{}
\end{align}
where $K_A = e^\frac{\Delta G_A}{2\kB T}$ 
\subsubsection{Linear-sweep voltammetry scaling}
\subsubsection{EIS scaling}
\end{document}


