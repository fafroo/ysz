%        File: EIS_capacitance.tex
%     Created: Fri Mar 13 10:00 am 2020 C
% Last Change: Fri Mar 13 10:00 am 2020 C
%
\documentclass[a4 paper]{article}
\usepackage[]{amsmath}
\usepackage{amsthm}
\usepackage[]{SIunits}
\usepackage[]{hyperref}
\newcommand{\dd}{\mathrm{d}}
\begin{document}

\section{Measurement}
A voltage protocol $U(t)$ is applied and a current response $I(t)$ is measured.
\section{Capacitance}
% For a certain voltage range $[-R,R]$ the following voltage protocol is given
% \begin{align}
%     U(t) = \text{linear triangular wave}
% \end{align}
% 
The capacitance $C$ is the change of the charge w.r.t.\ voltage.
For a sufficiently slow change of voltage U(t) we can write
\begin{align}
    I(t) = \frac{\dd Q}{\dd t} = \frac{\dd Q}{\dd U} \frac{\dd U}{\dd t} \approx C(U) v_{\text{rate}}~.
    \label{CAP01}
\end{align}

\section{Impedance spectroscopy}
For a certain frequency $\omega$ and voltage amplitude $A$ the protocol
\begin{align}
    U(t) = A \sin \left( \omega t \right) 
\end{align}
is applied. The measured current response is assumed to take form of
an amplified and shifted harmonic wave
\begin{align}
    I(t) = B \sin \left( \omega t  + \omega_0 \right) ~.
\end{align}

Using~\eqref{CAP01} for slow changes of the voltage ($\omega \ll 1$)
\begin{align}
    C(U=0) \approx \frac{I}{\frac{\dd U}{\dd t}} 
    &=  \frac{B \sin \left( \omega t+ \omega_0  \right)}
            {-A \omega\cos \left( \omega t  \right)}\\
            &\approx\left. \frac{B ( \omega t + \omega_0) }
                  {-A \omega \left( 1 - \omega t  \right)}\right|_{ t = 0}\\
    &= -\frac{B}{A} 
              \frac{\omega_0 }{\omega}   
\end{align}
yields a formula for approximation of the capacitance.
\end{document}


